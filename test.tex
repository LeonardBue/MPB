% Options for packages loaded elsewhere
\PassOptionsToPackage{unicode}{hyperref}
\PassOptionsToPackage{hyphens}{url}
%
\documentclass[
]{article}
\usepackage{amsmath,amssymb}
\usepackage{lmodern}
\usepackage{iftex}
\ifPDFTeX
  \usepackage[T1]{fontenc}
  \usepackage[utf8]{inputenc}
  \usepackage{textcomp} % provide euro and other symbols
\else % if luatex or xetex
  \usepackage{unicode-math}
  \defaultfontfeatures{Scale=MatchLowercase}
  \defaultfontfeatures[\rmfamily]{Ligatures=TeX,Scale=1}
\fi
% Use upquote if available, for straight quotes in verbatim environments
\IfFileExists{upquote.sty}{\usepackage{upquote}}{}
\IfFileExists{microtype.sty}{% use microtype if available
  \usepackage[]{microtype}
  \UseMicrotypeSet[protrusion]{basicmath} % disable protrusion for tt fonts
}{}
\makeatletter
\@ifundefined{KOMAClassName}{% if non-KOMA class
  \IfFileExists{parskip.sty}{%
    \usepackage{parskip}
  }{% else
    \setlength{\parindent}{0pt}
    \setlength{\parskip}{6pt plus 2pt minus 1pt}}
}{% if KOMA class
  \KOMAoptions{parskip=half}}
\makeatother
\usepackage{xcolor}
\IfFileExists{xurl.sty}{\usepackage{xurl}}{} % add URL line breaks if available
\IfFileExists{bookmark.sty}{\usepackage{bookmark}}{\usepackage{hyperref}}
\hypersetup{
  pdftitle={Heli-GPS Analysis in the Context of Mountain Pine Beetle},
  pdfauthor={Leonard Buechner},
  hidelinks,
  pdfcreator={LaTeX via pandoc}}
\urlstyle{same} % disable monospaced font for URLs
\usepackage[margin=1in]{geometry}
\usepackage{graphicx}
\makeatletter
\def\maxwidth{\ifdim\Gin@nat@width>\linewidth\linewidth\else\Gin@nat@width\fi}
\def\maxheight{\ifdim\Gin@nat@height>\textheight\textheight\else\Gin@nat@height\fi}
\makeatother
% Scale images if necessary, so that they will not overflow the page
% margins by default, and it is still possible to overwrite the defaults
% using explicit options in \includegraphics[width, height, ...]{}
\setkeys{Gin}{width=\maxwidth,height=\maxheight,keepaspectratio}
% Set default figure placement to htbp
\makeatletter
\def\fps@figure{htbp}
\makeatother
\setlength{\emergencystretch}{3em} % prevent overfull lines
\providecommand{\tightlist}{%
  \setlength{\itemsep}{0pt}\setlength{\parskip}{0pt}}
\setcounter{secnumdepth}{-\maxdimen} % remove section numbering
\ifLuaTeX
  \usepackage{selnolig}  % disable illegal ligatures
\fi
\usepackage[]{natbib}
\bibliographystyle{apa}

\title{Heli-GPS Analysis in the Context of Mountain Pine Beetle}
\author{Leonard Buechner}
\date{2022-03-16}

\begin{document}
\maketitle

\hypertarget{analysis-of-the-heli-gps-data}{%
\section{Analysis of the Heli-GPS
Data}\label{analysis-of-the-heli-gps-data}}

\hypertarget{introduction-and-data}{%
\subsection{Introduction and Data}\label{introduction-and-data}}

Most trees will have lost all needles, classifying as grey-attack, three
years after they have been attacked by the Mountain Pine Beetle (MPB).
When measuring LiDAR data at the end of March, al green-attack trees
that have been attacked in 2019 and before, as well as all red-attack
trees detected in 2020 and before, will now classify as grey attack. Of
those treees that have been detected as re-attack in 2021, some will
still classify as red-attack and others as grey-attack. This is due to
the fact that the heli-GPS data provided by the government of Alberta
cannot distinguish between different years of red-attack.

The roads are downloaded from several pages that I will have to cite
here. Flight locataions with the drone need to be accessable from the
road for efficient data collection. As the drone has to stay in line of
sight, it is not possible to fly far from the road. Reachable areas will
be defined through a buffer zone around paved and gravel roads as known
from \citep{GravelRoad23}.

\renewcommand\refname{References}
  \bibliography{references.bib}

\end{document}
